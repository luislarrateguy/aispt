\section{Prueba de meta}

El pseudocódigo de la prueba de meta es el siguiente:

\begin{verbatim}
Si energiaActual <= 0 entonces
   cumplio = falso
Sino
   convienePelear = (energiaActual - promedioPerdidaAlPelear) > 0
   convieneMoverse = (energiaActual - promedioPerdidaAlMoverse) > 0

   cumplio = (conoceTodoElTablero OR No convieneMoverse)
      && (No hayAlimentos AND (No hayEnemigos OR No convienePelear))
Fin Si

retornar cumplio
\end{verbatim}

Las variables \variable{convienePelear} y \variable{convieneMoverse} podrían
traducirse como ``tiene energía suficiente para pelear'' y ``tiene energía
suficiente para moverse''.

Notar el orden de evaluación de las operaciones booleanas. Las variables
\variable{hayAlimento} y \variable{hayEnemigo} sólo tienen sentido si se conoce
todo el ambiente. O sea que sólo serán evaluados si el agente conoce todo el
ambiente (o bien si no le conviene moverse).

