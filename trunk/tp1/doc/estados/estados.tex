\section{Estado del agente}

\subsection{Representación del estado}
\incluirimagen{scale=0.8}{estados/representacion_estado.pdf}

El \clase{Agente} posee un \clase{Estado}. Éste mantiene una visión del
ambiente, incompleta al principio y que se irá extendiendo con las percepciones
futuras.  \clase{EstadoCelda} es una enumeración. Su \variable{valor} puede ser
\emph{Desconocido}, \emph{Vacío}, \emph{Comida} o \emph{Enemigo}, según lo que
haya en la celda. Un \clase{Ambiente} posee 16 instancias de \clase{EstadoCelda}
que le indican, justamente, el estado de cada celda.

El método \metodo{hayEnemigo(int, int)} de \clase{Ambiente}\footnote{En la
figura no aparece por cuestiones de espacio, pero la clase completa fue
incluida en una imagen anterior, cuando se presentaba la arquitectura} permite
saber la existencia de un enemigo en una posición dada, y el método
\metodo{hayEnemigo()} es similar, pero al no recibir argumentos supone que se
pregunta para la posición actual del \clase{Agente}. Lo mismo sucede con
\metodo{hayComida(int, int)} y \metodo{hayComida()}. Por supuesto que si se
utiliza el primero para una posición que el Pacman aún no conoce, el resultado
será que ésta es \emph{Desconocida}.

\clase{Estado} posee un método \metodo{actualizar}, el cual recibe una
\clase{Percepción} que utiliza para extender su visión actual. Lo mismo pasa
con \clase{VisionAmbiente}. \clase{Estado} va manteniendo un promedio de la
energía perdida al pelear y moverse, y de la energía ganada al comer un
alimento. Estos valores son utilizados para decidir algunas acciones y también
son utilizados por la prueba de meta.

\subsection{Estado inicial}

\begin{itemize}
\item Cada celda del tablero (VisionAmbiente) inicializado en
\emph{Desconocida}.
\item Ubicado en una celda aleatoria del tablero mediante posición $X$ e $Y$.
Corresponde a la esquina superior izquierda, análogo con el sistema de
coordenadas de pantalla. La $X$ indica la columna (desplazamiento horizontal) y
la $Y$ la fila (desplazamiento vertical). Esta posición se almacena en la
variable \variable{posicionPacman} de la clase \clase{Ambiente}.
\item Agente.estado.energiaActual es menor o igual a 50.
\item Percepción de la celda superior, inferior, derecha e izquierda de su
posición actual. La Percepción le indica si esas celdas están en estado:
Vacía,Enemigo,Comida.
\item No percibe qué hay en la celda en la que inicialmente está parado.
\end{itemize}

\subsection{Estados intermedios (descripción general)}

\begin{itemize}
\item Agente.estado.getAmbiente().posicionPacman representa una posición
aleatoria.
\item Agente.estado.energiaActual tendrá como valor cualquier entero mayor a 0.
Incluso puede que sea mayor que la inicial y mayor que 50.
\item Conocimiento parcial del tablero.
\item Percepción de algunos enemigos y alimentos.
\end{itemize}

\subsection{Estado final}

\begin{itemize}
\item Ubicado en alguna celda del tablero, debido a la consecuencia de sus acciones..
\item Agente.estado.energiaActual es mayor a 0.
\item Todo el tablero es conocido.
\item Si tiene energía suficiente para pelear (según el promedio de pérdida en
luchas), no hay enemigos en el tablero. En caso contrario (al priorizar su
supervivencia) podrían quedar algunos de ellos en el mismo.
\item Si tiene energía suficiente para moverse (según el promedio de pérdida
por movimiento), no hay alimentos en el tablero. En caso contrario (al
priorizar su supervivencia) podrían quedar alimentos presentes.
\end{itemize}
