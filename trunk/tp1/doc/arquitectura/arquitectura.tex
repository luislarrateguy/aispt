\section{Arquitectura}

\begin{figure}[h]
\incluirimagen{scale=0.50}{arquitectura/arquitectura.png}
\caption{Arquitectura del agente}
\label{fig:arquitectura}
\end{figure}

La figura~\ref{fig:arquitectura} es un dibujo simple de la arquitectura. Es
similar a la que está de ejemplo en el enunciado. Ésta indica que el simulador
le entrega al agente una percepción inicial. El agente toma sus decisiones y
devuelve una acción. Esta acción modifica al ambiente real actual. Luego el
simulador vuelve a entregarle otra percepción, y el agente retorna nuevamente
una acción. Éste ciclo continúa.

\begin{figure}
\incluirimagen{scale=0.50}{arquitectura/agente.pdf}
\caption{\clase{Agente}}
\label{fig:clases_agente}
\end{figure}

\begin{figure}
\incluirimagen{scale=0.55}{arquitectura/acciones.pdf}
\caption{Acciones}
\label{fig:clases_acciones}
\end{figure}

\begin{figure}
\incluirimagen{scale=0.60}{arquitectura/ambiente.pdf}
\caption{Jerarquía de \clase{Ambiente}}
\label{fig:clases_ambiente}
\end{figure}

\begin{figure}
\incluirimagen{scale=0.60}{arquitectura/busqueda.pdf}
\caption{Algoritmos de búsqueda}
\label{fig:clases_busqueda}
\end{figure}

\begin{figure}
\incluirimagen{scale=0.60}{arquitectura/relaciones_agente.pdf}
\caption{Relaciones del Agente con otras clases}
\label{fig:clases_relaciones_agente}
\end{figure}

\begin{figure}
\incluirimagen{scale=0.60}{arquitectura/simulador.pdf}
\caption{Relaciones del Simulador con otras clases}
\label{fig:clases_relaciones_simulador}
\end{figure}

En el diagrama de clases, que aparece en varias figuras debido a su tamaño
(figura~\ref{fig:clases_agente}, \ref{fig:clases_acciones},
\ref{fig:clases_ambiente}, \ref{fig_clases_busqueda},
\ref{fig:clases_relaciones_agente} y \ref{fig:clases_relaciones_simulador})
encontramos más detalles. Podemos observar que el agente almacena un estado
propio del ambiente. A diferencia del real, éste se irá completando a medida
que el agente lo recorra. También podemos notar que las acciones poseen un
método \metodo{ejecutar} que recibe como parámetro a una instancia de alguna
clase que extienda a \clase{Ambiente} (o sea, una instancia de
\clase{AmbienteReal} o de \clase{VisionAmbiente}) para poder realizar sus
modificaciones.

La clase abstracta \clase{Busqueda} es la que se encarga de expandir nodos y
agregarlos a la cola. Las subclases implementan lo específico de cada
estrategia: el cálculo del valor de la función de evaluación.

