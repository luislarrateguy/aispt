\subsection{Código Prolog}

El archivo \emph{base\_conocimiento.pl} contiene las reglas necesarias para el
proceso de inferencia del Agente. Inicialmente consultando desde Java por \linebreak
\emph{mejorAccion(X,S)} se obtenía la mejor acción X dada una situación S. Esta
forma de consultar y con reglas solamente deductivas, hacía todo el
procesamiento muy lento (por cada consulta se hacían llamadas recursivas en las
reglas de estado sucesor).

Para evitar esto se utilizó la familia de predicados \emph{assert}.
Creamos un predicado \emph{est(S1)} que calcula el estado para una situación
S1 en base a una anterior. Esto nos trajo aparejado otro problema, que era
que cada \emph{assert} incluido en cada uno de los predicados \emph{est(S)} 
generaban duplicados en la base de conocimiento (debido a que se podía estar
 percibiendo una celda que además se estaba heredando su estado de la situación
 anterior). 

Todas estas cuestiones (y otras) las resolvimos definiendo como se encuentran
 actualmente los predicados \emph{est(S1)} y llamándolos explícitamente desde
 Java mediante \emph{findall(\_,est(S1),\_)} con el fin de que todos unifiquen (lo que hace es buscar todas las soluciones posibles, logrando que los axiomas de estado sucesor unifiquen la cantidad de veces necesarias).
