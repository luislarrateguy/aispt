\subsection{Importar el código a Eclipse, compilarlo y ejecutarlo\label{sec:inst}}

Tanto para GNU/Linux como para Windows, hemos creado un paquete personalizado
de los binarios de SWI-Prolog, quitando los componentes que no son necesarios
del paquete oficial.

La dirección de descarga es
\href{http://code.google.com/p/aispt/downloads/list}{http://code.google.com/p/aispt/downloads/list}.
El archivo con los binarios de SWI-Prolog para
GNU/Linux se llama \linebreak\archivo{pl-5.6.35-bin-linux.tar.bz2}, y el
correspondiente para Windows es \linebreak\archivo{pl-5.6.35-bin-win.zip}

% GNU/Linux
\subsubsection{En GNU/Linux}
\begin{itemize}

\item En Eclipse, crear un proyecto Java. Debe utilizarse Java 5.0.

\item Importar en el proyecto antes creado el archivo \archivo{tp2-ia.zip}
(General/Archive file).  Cuando pregunte por sobreescribir los archivos
\archivo{.project} y \archivo{.classpath} responder afirmativamente.

\item Hacer clic derecho en la carpeta del proyecto, ir al menú \emph{Run As},
y elegir la opción \emph{Run...}

\item En el panel de la izquierda, hacer doble clic sobre \emph{Java Application}.
para crear una configuración nueva.

\item En la solapa \emph{Main}, elegir como \emph{Main class} la clase \variable{tpsia.tp2.Main}.

\item En la solapa \emph{Arguments}, en la caja de texto \emph{Program arguments} colocar:
``logger.config'' (sin las comillas). Es el archivo de configuración de log4j. En la caja de texto \emph{VM arguments}
colocar:\newline``-Djava.library.path=\$\{SWI\_PROLOG\_INSTALL\_DIR\}/lib/i386-linux''.\newline
Reemplazar \$\{SWI\_PROLOG\_INSTALL\_DIR\} por el path absoluto donde fue descomprimido SWI-Prolog desde los
archivos comprimidos que hemos mencionado antes. Vale aclarar que es bueno colocar la ruta entre comillas
dobles, por si la misma tiene espacios, ya que de otra forma no funcionaría.

\item En la solapa \emph{Environment}, crear una nueva variable de entorno con el nombre ``SWI\_HOME\_DIR'',
y valor ``\$\{SWI\_PROLOG\_INSTALL\_DIR\}''.

\item Clic en el botón \emph{Apply}.

\item Clic en el botón \emph{Run}. La salida por consola muestra un dibujo de
la visión del ambiente del pacman, las decisiones que va tomando, su posición,
energía, etc.

\end{itemize}


% Windows
\subsubsection{En Windows}

Los pasos son muy similares que en GNU/Linux, con la salvedad de que:

\begin{itemize}

\item En la solapa \emph{Arguments}, dentro de la caja de texto \emph{VM
arguments}, al configurar cómo correr el trabajo práctico, se debe
colocar:\newline ``-Djava.library.path=\$\{SWI\_PROLOG\_INSTALL\_DIR\}/bin'',
ya que es en la carpeta \archivo{bin} donde se encuentran las librerías en la
distribución para Windows de SWI-Prolog. Aquí se hace más énfasis en colocar la
ruta entre comillas dobles si ésta tiene espacios en blanco.

\item En la solapa \emph{Environment} hay que agregar una variable con nombre
``PATH'' en lugar de ``SWI\_HOME\_DIR'', ya que esta última no parece ser
tomada en Windows. El valor de la misma es:\newline
``-Djava.library.path=\$\{SWI\_PROLOG\_INSTALL\_DIR\}/bin''

\item Finalmente, se aplican los cambios presionando en \emph{Apply}.

\end{itemize}

