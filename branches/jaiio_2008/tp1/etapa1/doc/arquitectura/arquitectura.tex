\section{Arquitectura}

En esta entrega incluimos dos imágenes en formato PNG. Una es un dibujo simple
de la arquitectura. Es similar a la que está de ejemplo en el enunciado. Esta
indica que el simulador le entrega al agente una percepción inicial. El agente
toma sus decisiones y devuelve una acción. Esta acción modifica al ambiente
real actual. Luego el simulador vuelve a entregarle otra percepción, y el
agente retorna nuevamente una acción. Éste ciclo continúa.

En el diagrama de clases encontramos más detalles. Podemos observar que el
agente almacena un estado propio del ambiente. A diferencia del real, éste se
irá completando a medida que el agente lo recorra. También podemos notar que
las acciones poseen un método \metodo{ejecutar} que recibe como parámetro a una
instancia de \clase{Ambiente} para poder realizar sus modificaciones.
