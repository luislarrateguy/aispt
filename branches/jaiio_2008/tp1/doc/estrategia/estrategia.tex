\section{Estrategia}

Hemos escogido A* como estrategia primaria para nuestro agente.  Si bien además
están implementadas las estrategias de \emph{Profundidad}, \emph{Amplitud},
\emph{Costo Uniforme} y \emph{Avara}, escogimos A* porque es la que mejor
parecía resolver los problemas en la mayoría de los casos.

Cada acción tiene un costo. El avanzar a cualquier dirección supone un aumento
de 60. Pelear tiene un costo de 40, y Comer 20.

La función de evaluación $f(n)$ de la estrategia A* es igual a $g(n) + h(n)$.
En nuestra implementación:

\begin{itemize}
\item $g(n) = costoDelNodoPadre + costoAccionGeneradora$
\item $h(n) = cantidad de celdas desconocidas$.
\end{itemize}

Para que una heurística sea consistente, se debe cumplir que $h(n) <= c(n,a,n')
+ h(n')$, siendo \emph{n'} un nodo que se alcanza desde el nodo \emph{n} al
aplicar la acción \emph{a}.

Moviendo $h(n')$ al término de la izquierda, queda $c(n,a,n') >= h(n) - h(n')$.
La máxima diferencia posible del segundo término es 4, pero el menor costo de
cualquier acción es 20. Por lo tanto la heurística es consistente, y por lo
tanto también admisible.

