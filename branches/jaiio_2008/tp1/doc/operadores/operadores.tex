\section{Operadores de búsqueda}

La búsqueda se efectúa operando sobre un estado inicial, aplicando sobre éste
todos los operadores en los cuales su precondición sea verdadera, obteniendo
nuevos estados. El estado se almacena dentro de la variable privada
\variable{estado} del \clase{Agente}.

\begin{itemize}

\item \textbf{AvanzarArriba}\newline
\underline{Precondición:} (energiaActual - energiaPerdidaPromedioAlAvanzar) > 0
\newline
\underline{Resultado:} El Pacman es desplazado a su celda superior (movimiento
sobre una esfera\footnote{Es decir, si se encuentra en la primer fila y se
dirige hacia arriba, aparecerá en la misma columna pero en la fila inferior del
tablero.  Esto es similar para los demás operadores de movimiento.})

\item \textbf{AvanzarAbajo}\newline
\underline{Precondición:} (energiaActual - energiaPerdidaPromedioAlAvanzar) > 0
\newline
\underline{Resultado:} El Pacman es desplazado a su celda inferior (movimiento
sobre una esfera)\newline

\item \textbf{AvanzarDerecha}\newline
\underline{Precondición:} (energiaActual - energiaPerdidaPromedioAlAvanzar) > 0
\newline
\underline{Resultado:} El Pacman es desplazado a su celda derecha (movimiento
sobre una esfera)\newline

\item \textbf{AvanzarIzquierda}\newline
\underline{Precondición:} (energiaActual - energiaPerdidaPromedioAlAvanzar) > 0
\newline
\underline{Resultado:} El Pacman es desplazado a su celda izquierda (movimiento
sobre una esfera)\newline

\item \textbf{Comer}\newline
\underline{Precondición:} \metodo{est.getAmbiente().hayComida(x,y)} = true;
siendo $x$ e $y$ la posición actual.

\begin{verbatim}
Si est.getAmbiente().hayComida(x,y)
   -> visionAmbiente.comer()
\end{verbatim}

\underline{Resultado:} \metodo{visionAmbiente.comer()}. El Pacman ``come''.
Alimento en posicionActual desaparece y la celda queda en estado \textit{Vacia}

\item \textbf{Pelear}\newline
\underline{Precondición:} \metodo{est.getAmbiente().hayEnemigo(x,y) AND}
\newline
\metodo{(energiaActual - energiaPromedioPerdidaEnPelea) > 0};
\newline
siendo X e Y la posición actual.

\begin{verbatim}
Si est.getAmbiente().hayEnemigo(x,y)
   -> visionAmbiente.pelear()
\end{verbatim}

\underline{Resultado:} \metodo{visionAmbiente.pelear()}. El Pacman ``pelea''.
Enemigo en posicionActual desaparece y la celda queda en estado \textit{Vacia}

\end{itemize}
