\section{Notas y aclaraciones}

El directorio \emph{bin} posee los archivos necesarios para probar el
simulador.  Simplemente hay que ubicarse en el mismo y ejecutar el siguiente
comando: \makebox{\texttt{java -jar simulador.jar}.}

Todo el código fuente se encuentra en el archivo mencionado
\linebreak\emph{bin/simulador.jar}, así como las librerías utilizadas
(\emph{calculador.jar} y \linebreak\emph{log4j-1.2.13.jar}) y los archivos
\emph{.project} y \emph{.classpath}. Por lo tanto sería posible importar el
proyecto desde Eclipse a partir de éste archivo.

El directorio \emph{ejecuciones} tiene un archivo XML y un PDF. Ambos se
refieren al árbol búsqueda del agente antes de tomar la primera decisión. El
archivo PDF tiene un dibujo del árbol generado, dividido en subárboles por
cuestiones de espacio. En el mismo hay algunas indicaciones para su lectura.

El archivo \emph{bin/logger.config} puede ser editado para que el simulador
genere nuevamente el archivo XML o el archivo \LaTeX~para luego generar el PDF
(es necesario contar con algunos paquetes extra).

Además de la estrategia A*, la cual hemos escogido para nuestro trabajo
práctico, nosotros hemos implementado, aunque sin demasiado debugging, las
demás: amplitud, profundidad, costo uniforme y avara. Ver el código fuente, en
el paquete \emph{tp1.busqueda}.

