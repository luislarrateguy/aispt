\section{Estrategia}

Como estrategia primaria hemos escogido para nuestro agente A*.
Si bien además están implementadas las estrategias de Profundidad, Amplitud
CostoUniforme y Avara, escogimos A* porque es la que mejor parecía resolver los 
problemas en la mayoría de los casos.
El costo de aplicar un operador Avanzar es de 50, el de aplicar Pelear de 38, y el 
de aplicar Comer es de 20.
Por lo tanto
g(n) = costoDelPadre + costoAccion

Para A* utilizamos la siguiente heurística:
sea 
ccd: la cantidad de celdas desconocidas
cev: la cantidad de enemigos vivos
cas: la cantidad de alimento sin comer

h(n) = ccd + cev +cas

La cantidad de celdas desconocidas tendrá un máximo de 12 (16 - las 4 iniciales), 
y un mínimo de 0.
La cantidad de enemigos vivos + la cantidad de comida sin comer tendrá un máximo 
de 16 y un mínimo de 0.

f(n) = g(n) + h(n)

Notar que:
* La suma de enemigos y comida nunca puede superar 16 (cantida de celdas)
* La cantidad de celdas vacias inicialmente es 12, en el primer movimiento se decrementa 
en 4, y luego como máximo en 2 (debido a que e sun tablero de 4x4)
* Con cada acción comer o pelear sólo disminuye en 1 la cantidad de celdas
con enemigo o alimento.

la heurística elegida es consistente (y admisible) ya que:
MAX(h(n)) = 16    // Corresponde a 12 celdas desconocidas y 4 elementos alrededor.
MIN(h(n)) = 0     // objetivo
MIN(g(n)) =  50   // costo del primer nivel de nodos, debido a la primer accion avanzar

h(n) <= g(n') + h(n')

En el primer caso

MAX(h(n')) <= MIN(g(n')) + (MAX(h(n')) - 4)

Ya que moverse es su única opción al comienzo
De ahí en más

h(n) <= g(n') + h(n')

donde h(n) >= h(n')
y g(n') > h(n) (debido a los mínimos y máximos)

Hemos escogido la estrategia de costo uniforme para nuestro agente. Un nodo se
descarta si se pronostica la muerte del Pac-Man. Si todos los nodos implican su
muerte, entonces el agente finaliza.

La función f(n) prioriza comer primero los alimentos (les daría un ``mayor
valor'' a los caminos que realizan esta accion). También se encargaría de premiar
a aquellos caminos que hacen que el agente conozca más el tablero.
