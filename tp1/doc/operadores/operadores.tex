\section{Operadores de búsqueda}

La búsqueda se efectua operando sobre un estado inicial y se le va 
operadores, obteniendo nuevos estados. El estado se almacena dentro de
la variable privada \variable{estado} del tipo \clase{Estado}

********************comentario ****************no me anduvo fixme ****
Ponemos la precondición para cada Avanzar que !hayComida(x,y). ?
Me parece que seria una buena optimizacion
********************comentario ****************no me anduvo fixme ****

\begin{itemize}

\item \textbf{AvanzarArriba}\newline
\underline{Precondicion:} energiaActual > energiaPerdidaPromedioEnPelea\newline
\underline{Resultado:} El Pacman es dezplazado a su celda superior (movimiento
sobre una esfera\footnote{Es decir, si se encuentra en la primer fila y se
dirige hacia arriba, aparecerá en la misma columna pero en la fila de abajo.
Esto es similar para los demás operadores de movimiento.})

\item \textbf{AvanzarAbajo}\newline
\underline{PreCondicion:}energiaActual > energiaPerdidaPromedioEnPelea
\underline{Resultado:} El Pacman es dezplazado a su celda inferior (movimiento
sobre una esfera)\newline

\item \textbf{AvanzarDerecha}\newline
\underline{PreCondicion:} energiaActual > energiaPerdidaPromedioEnPelea\newline
\underline{Resultado:} El Pacman es dezplazado a su celda derecha (movimiento
sobre una esfera)\newline

\item \textbf{AvanzarIzquierda}\newline
\underline{PreCondicion:} energiaActual > energiaPerdidaPromedioEnPelea\newline
\underline{Resultado:} El Pacman es dezplazado a su celda izquierda (movimiento
sobre una esfera)\newline

\item \textbf{Comer}\newline
\underline{PreCondicion:} \metodo{est.getAmbiente().hayComida(x,y)} = true;
siendo X e Y la posición actual.

\begin{verbatim}
Si est.getAmbiente().hayComida(x,y)
   -> visionAmbiente.comer()
\end{verbatim}

\underline{Resultado:} \metodo{visionAmbiente.comer()}. El Pacman ``come''.
Alimento en posicionActual desaparece y la celda queda en estado \textit{Vacia}


\item \textbf{Pelear}\newline
\underline{PreCondicion:} \metodo{est.getAmbiente().hayEnemigo(x,y) AND 
energiaActual > energiaPromedioPerdidaEnPelea} = true;
siendo X e Y la posición actual.

\begin{verbatim}
Si est.getAmbiente().hayEnemigo(x,y)
   -> visionAmbiente.pelear()
\end{verbatim}

\underline{Resultado:} \metodo{visionAmbiente.pelear()}. El Pacman ``pelea''.
Enemigo en posicionActual desaparece y la celda queda en estado \textit{Vacia}

\end{itemize}
