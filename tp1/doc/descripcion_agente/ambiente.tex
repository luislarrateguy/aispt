\subsection{Ambiente}

El ambiente del agente es el tablero de juego. El mismo es:

\begin{itemize}
\item \textbf{Parcialmente observable:} Ya que el agente no tiene acceso al
estado completo de su ambiente en cada instante.

\item \textbf{Estocástico:} Si el agente decide pelear contra un enemigo, no
puede saber de antemano si el hacerlo provocará que muera. Por lo tanto el
nuevo estado del ambiente no está definido por el estado actual más la acción
que el agente lleve acabo.

\item \textbf{Episódico:} Al percibir el ambiente, el agente toma una decisión
totalmente independiente de las decisiones anteriores.

\item \textbf{Estático:} El ambiente del agente (el tablero) no cambia. Los
enemigos y los alimentos aparecen en una posición dada al principio del juego,
y ésta permanece fija hasta el final.

\item \textbf{Discreto:} El conjunto de percepciones y acciones es finito.

\item \textbf{Agente individual:} Ya que el Pac-Man es el único agente en
juego.  No hay otros.

\end{itemize}

