%
% Realizado con Software Libre
% 
% Usen esta página para hacerles saber a los lectores de su documento,
% que el mismo esta desarrollado íntegramente con Software de Código
% Abierto
%
% Versión 1.0.1 - 23/Septiembre/2006
% Por Milton Pividori

\thispagestyle{empty}

\begin{center}
\LARGE{Hecho con Software Libre}
\end{center}

\noindent

El trabajo práctico, así como este documento, fueron realizados íntegramente
usando Software Libre. Las herramientas utilizadas fueron:

\begin{flushleft}
\begin{itemize}

% LaTeX
\item \textbf{\LaTeX} - \href{http://www.latex-project.org/}
  {http://www.latex-project.org/}
\linebreak\LaTeX{} es un sistema de preparación de documentos. Éste
  documento esta hecho con él.

% Eclipse
%\item \textbf{Eclipse} - \href{http://www.eclipse.org/}
%  {http://www.eclipse.org/}

% ArgoUML
\item \textbf{ArgoUML} - \href{http://argouml.tigris.org/}
  {http://argouml.tigris.org/}
\linebreak Herramienta de modelado UML. Es multiplataforma (esta desarrollado
en Java) y está disponible en 10 idiomas.

% Subversion
\item \textbf{Subversion} - \href{http://subversion.tigris.org/}
  {http://subversion.tigris.org/}
\linebreak Subversion es un sistema de control de versiones diseñado
  específicamente para reemplazar a CVS. Lo utilizamos para versionar
  tanto el código fuente del evaluador como el de éste documento.

% Subclipse
%\item \textbf{Subclipse} - \href{http://subclipse.tigris.org/}
%{http://subclipse.tigris.org/}
%\linebreak Subclipse es un plugin para Eclipse que integra Subversion a este
%ultimo.

% JUnit
%\item \textbf{JUnit} - \href{http://www.junit.org/}
%  {http://www.junit.org/}
%\linebreak Framework para unit testing. Lo utilizamos para verificar el correcto
%funcionamiento individual de cada clase.

% Ubuntu Linux
\item \textbf{Ubuntu Linux} - \href{http://www.ubuntu.com/}
  {http://www.ubuntu.com/}
\linebreak Basada en Debian GNU/Linux, Ubuntu es un sistema operativo
enteramente de fuente abierta, y uno de los más populares hoy en día.

% GNU Aspell
\item \textbf{GNU Aspell} - \href{http://aspell.sourceforge.net//}
  {http://aspell.sourceforge.net/}
\linebreak GNU Aspell es una utilidad para chequear la ortografía.

% Vim
\item \textbf{Vim} - \href{http://www.vim.org/}
  {http://www.vim.org/}
\linebreak Vim es un editor de texto avanzado. Hay disponible una
versión gráfica para Windows.

\end{itemize}
\end{flushleft}

\begin{center}
\includegraphics[scale=0.50]{logofsf.jpg}
\linebreak\href{http://www.fsf.org}{http://www.fsf.org}
\end{center}
