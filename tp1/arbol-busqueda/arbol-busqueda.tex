\documentclass[a0,landscale]{a0poster}
\usepackage{mathptmx}
\usepackage[scaled=.90]{helvet}
\usepackage{courier}
\usepackage{qtree}
\usepackage{nodo}
\usepackage[spanish]{babel}
\usepackage[utf8]{inputenc}
\title{Árbol de ejecución - Estrategia: A*}
\author{}
\begin{document}
\maketitle

\section{Notas}
Este es el árbol con los 16 primeros nodos elegidos por el algoritmo de
búsqueda. También se muestran los nodos hijos de los mismos.

La forma de leer la secuencia es de arriba hacia abajo, de izquierda a derecha.
Por lo tanto el primero nodo tomado es aquel con Id = 1, luego aquel con Id =
2, luego el 5, el 3, el 4, el 19, etc.

$f$ representa la función de evaluación, igual a $g(n) + h(n)$. En la
documentación principal hay mas información sobre esto en la sección
\emph{Estrategia}.

El campo \emph{A} indica la acción que generó el nodo.

\section{Secuencia de subárboles}

\begin{figure}[!h]
\Tree [.\nodo{1}{0.0}{-} \nodo{2}{68.0}{arriba} \nodo{3}{68.0}{derecha} \nodo{4}{68.0}{abajo} \nodo{5}{68.0}{izquierda} ]
\Tree [.\nodo{2}{68.0}{arriba} \nodo{6}{116.0}{pelear} \nodo{7}{134.0}{arriba} \nodo{8}{134.0}{derecha} \nodo{9}{136.0}{abajo} \nodo{10}{134.0}{izquierda} ]
\Tree [.\nodo{5}{68.0}{izquierda} \nodo{11}{134.0}{arriba} \nodo{12}{136.0}{derecha} \nodo{13}{134.0}{abajo} \nodo{14}{134.0}{izquierda} ]
\Tree [.\nodo{3}{68.0}{derecha} \nodo{15}{134.0}{arriba} \nodo{16}{134.0}{derecha} \nodo{17}{134.0}{abajo} \nodo{18}{136.0}{izquierda} ]
\end{figure}
\begin{figure}[!h]
\Tree [.\nodo{4}{68.0}{abajo} \nodo{19}{136.0}{arriba} \nodo{20}{134.0}{derecha} \nodo{21}{134.0}{abajo} \nodo{22}{134.0}{izquierda} ]
\Tree [.\nodo{6}{116.0}{pelear} \nodo{23}{182.0}{arriba} \nodo{24}{182.0}{derecha} \nodo{25}{184.0}{abajo} \nodo{26}{182.0}{izquierda} ]
\Tree [.\nodo{22}{134.0}{izquierda} \nodo{27}{198.0}{arriba} \nodo{28}{200.0}{derecha} \nodo{29}{198.0}{abajo} \nodo{30}{198.0}{izquierda} ]
\Tree [.\nodo{10}{134.0}{izquierda} \nodo{31}{198.0}{arriba} \nodo{32}{200.0}{derecha} \nodo{33}{198.0}{abajo} \nodo{34}{198.0}{izquierda} ]
\end{figure}
\begin{figure}[!h]
\Tree [.\nodo{11}{134.0}{arriba} \nodo{35}{198.0}{arriba} \nodo{36}{198.0}{derecha} \nodo{37}{200.0}{abajo} \nodo{38}{198.0}{izquierda} ]
\Tree [.\nodo{21}{134.0}{abajo} \nodo{39}{200.0}{arriba} \nodo{40}{198.0}{derecha} \nodo{41}{198.0}{abajo} \nodo{42}{198.0}{izquierda} ]
\Tree [.\nodo{7}{134.0}{arriba} \nodo{43}{198.0}{arriba} \nodo{44}{198.0}{derecha} \nodo{45}{200.0}{abajo} \nodo{46}{198.0}{izquierda} ]
\Tree [.\nodo{13}{134.0}{abajo} \nodo{47}{200.0}{arriba} \nodo{48}{198.0}{derecha} \nodo{49}{198.0}{abajo} \nodo{50}{198.0}{izquierda} ]
\end{figure}
\begin{figure}[!h]
\Tree [.\nodo{15}{134.0}{arriba} \nodo{51}{198.0}{arriba} \nodo{52}{198.0}{derecha} \nodo{53}{200.0}{abajo} \nodo{54}{198.0}{izquierda} ]
\Tree [.\nodo{14}{134.0}{izquierda} \nodo{55}{198.0}{arriba} \nodo{56}{200.0}{derecha} \nodo{57}{198.0}{abajo} \nodo{58}{198.0}{izquierda} ]
\Tree [.\nodo{8}{134.0}{derecha} \nodo{59}{198.0}{arriba} \nodo{60}{198.0}{derecha} \nodo{61}{198.0}{abajo} \nodo{62}{200.0}{izquierda} ]
\Tree [.\nodo{16}{134.0}{derecha} \nodo{63}{198.0}{arriba} \nodo{64}{198.0}{derecha} \nodo{65}{198.0}{abajo} \nodo{66}{200.0}{izquierda} ]
\end{figure}


\end{document}
