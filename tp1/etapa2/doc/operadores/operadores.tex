\section{Operadores de búsqueda}
La búsqueda se efectua operando sobre un estado inicial y se le va aplicando operadores, obteniendo nuevos estados.\newline
El estado se almacena dentro de la variable privada \textit{est} del tipo \textit{Estado}\newline
\TODO {Cómo explico una precondición "no tiene"?}

\begin{itemize}

\item \textbf{MoverArriba}\newline
\underline{Precondicion:} no tiene.\newline
\underline{Resultado:} El Pacman es dezplazado a su celda superior (movimiento
sobre una esfera\footnote{Es decir, si se encuentra en la primer fila y se
dirige hacia arriba, aparecerá en la misma columna pero en la fila de abajo.
Esto es similar para los demás operadores de movimiento.})

\item \textbf{MoverAbajo}\newline
\underline{PreCondicion:}  no tiene.\newline
\underline{Resultado:} El Pacman es dezplazado a su celda inferior (movimiento
sobre una esfera)

\item \textbf{MoverDerecha}\newline
\underline{PreCondicion:}  no tiene.\newline
\underline{Resultado:} El Pacman es dezplazado a su celda derecha (movimiento
sobre una esfera)

\item \textbf{MoverIzquierda}\newline
\underline{PreCondicion:}  no tiene.\newline
\underline{Resultado:} El Pacman es dezplazado a su celda izquierda (movimiento
sobre una esfera)

\item \textbf{Comer}\newline
\underline{PreCondicion:} \metodo{est.getAmbiente().hayComida(x,y)} = true; siendo X e Y la posición actual.\newline
\underline{Resultado:} \metodo{visionAmbiente.comer()}. El Pacman ``come''. Alimento en posicionActual desaparece
y la celda queda en estado \textit{Vacia}

\item \textbf{Pelear}\newline
\underline{PreCondicion:} \metodo{est.getAmbiente().hayEnemigo(x,y)} = true; siendo X e Y la posición actual\newline
\underline{Resultado:} \metodo{visionAmbiente.pelear()}. El Pacman ``pelea''. Enemigo en posicionActual desaparece
y la celda queda en estado \textit{Vacia}

\end{itemize}
