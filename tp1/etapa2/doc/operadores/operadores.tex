\section{Operadores de búsqueda}
\begin{itemize}

\item \textbf{MoverArriba}\newline
\underline{Precondicion:} Siempre puede MoverArriba.\newline
\underline{Resultado:} El Pacman es dezplazado a su celda superior (movimiento
sobre una esfera\footnote{Es decir, si se encuentra en la primer fila y se
dirige hacia arriba, aparecerá en la misma columna pero en la fila de abajo.
Esto es similar para los demás operadores de movimiento.})

\item \textbf{MoverAbajo}\newline
\underline{PreCondicion:} Siempre puede MoverAbajo.\newline
\underline{Resultado:} El Pacman es dezplazado a su celda inferior (movimiento
sobre una esfera)

\item \textbf{MoverDerecha}\newline
\underline{PreCondicion:} Siempre puede MoverDerecha.\newline
\underline{Resultado:} El Pacman es dezplazado a su celda derecha (movimiento
sobre una esfera)

\item \textbf{MoverIzquierda}\newline
\underline{PreCondicion:} Siempre puede MoverIzquierda.\newline
\underline{Resultado:} El Pacman es dezplazado a su celda izquierda (movimiento
sobre una esfera)

\item \textbf{Comer}\newline
\underline{PreCondicion:} el método \metodo{hayComida} devuelve \emph{true} en la
posicionActual.\newline
\underline{Resultado:} El Pacman ``come''. Alimento en posicionActual desaparece
y la celda queda en estado \textit{Vacia}

\item \textbf{Pelear}\newline
\underline{PreCondicion:} el método \metodo{hayEnemigo} en
posicionActual.\newline
\underline{Resultado:} El Pacman ``pelea''. Enemigo en posicionActual desaparece
y la celda queda en estado \textit{Vacia}

\end{itemize}
