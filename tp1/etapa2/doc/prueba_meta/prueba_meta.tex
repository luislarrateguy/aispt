\section{Prueba de meta}

Como desconocemos un poco el funcionamiento del agente, hay algunas dudas que
nos hacen suponer cosas.

El pseudocódigo de la prueba de meta es el siguiente:

\begin{verbatim}
Si visionAmbiente.ConocimientoTotal entonces

  convienePelear = energiaPromedioPelea < energiaActual
  convieneMoverse = energiaPromedioMover < energiaActual

  Si ( (No visionAmbiente.HayAlimentosSinComer) OR
      (visionAmbiente.HayAlimentosSinComer AND No convieneMoverse) )
    AND ( (No visionAmbiente.HayEnemigos) OR
      (visionAmbiente.HayEnemigos AND No convienePelear) )

  retornar EXITO

Sino
  Si energiaInsuficienteParaMover retornar EXITO

retornar FRACASO
\end{verbatim}

Propongo esta prueba de meta (por mas que se implemente disinto, pero para que se lea mejor. En vez de poner la negacion al principio, aprovechar que es pseudo-codigo y usar un metodo para cada cosa (aunque uno sea negacion del otro)
\begin{verbatim}

Si ConoceTodoElAmbiente entonces
  ConvienePelear = energiaPromedioPelea < energiaActual
  ConvieneMoverse = energiaPromedioMover < energiaActual
  Si (ConvieneMoverse AND (HayAlimentosSinComer OR 
                          (HayEnemigos AND ConvienePelear) 
     )  
     retornar FRACASO
  Sino
     retornet EXITO
Sino
  Si convieneMoverse entonces
     retornar FRACASO  // Se da por vencido. Asume exito en su objetivo, porque quiere sobrevivir!
  Sino
     retornar EXITO

\end{verbatim}

