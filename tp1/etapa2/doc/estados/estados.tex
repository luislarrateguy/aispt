\section{Estado del agente}

\subsection{Representación del estado}
\incluirimagen{scale=0.7}{estados/diagramaClases.png}

\subsection{Estado inicial}

\begin{itemize}
\item Cada celda del tabler (VisionAmbiente) inicializado en \textit{Desconocida}
\item Ubicado en una celda aleatoria del tablero mediante posición $X$ e $Y$. Corresponde a la esquina superior izquierda, análogo con el sistema de coordenadas de pantalla. La $X$ indica la columna (desplazamiento horizontal) y la $Y$ la fila (desplazamiento vertical).
\item Energía inicial: menor o igual a 50.
\item Percepción de la celda superior, inferior, derecha e izquierda de su posición actual. La Percepción le indica si esas celdas están en estado: Vacía,Enemigo,Comida.
\item No percibe qué hay en la celda en la que inicialmente está parado.
\end{itemize}

\subsection{Estados intermedios (descripción general)}

\begin{itemize}
\item Ubicado en una celda aleatoria del tablero mediante posición $X$ e $Y$.
\item Energía: La energía tendrá como valor cualquier entero mayor a 0. Incluso puede que sea mayor que la inicial y mayor que 50.
\item Conocimiento parcial del tablero.
\item Percepción de algunos enemigos y alimentos.
\end{itemize}

\subsection{Estado final}

\begin{itemize}
\item Ubicado en alguna celda del tablero, debido a la consecuencia de sus acciones..
\item Energía mayor o igual a 0.
\item Todo el tablero es conocido.
\item Si tiene energía suficiente para pelear (según el promedio de pérdida en
luchas), no hay enemigos en el tablero. En caso contrario (al priorizar su
supervivencia) podrían quedar algunos de ellos en el mismo.
\item Si tiene energía suficiente para moverse (según el promedio de pérdida
por movimiento), no hay alimentos en el tablero. En caso contrario (al priorizar
su supervivencia) podrían quedar alimentos presentes.
\end{itemize}
