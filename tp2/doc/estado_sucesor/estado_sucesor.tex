\section{Axiomas de estado sucesor}

\Nota{Ahora lo que era antes \emph{Resultado(accion,s)} es \emph{AccionEjecutada(acción,s)},
e indica, como su nombre sugiere, la acción que fue ejecutada en la situación \emph{s}.}

\begin{itemize}

\item $\forall x,y$: posición(x,y,s+1) $\Leftrightarrow$
\newline (AccionEjecutada(Mover(arriba),s) $\land$ posicion(x, SumarPosicion(y,-1),s))
\newline $\lor$
\newline (AccionEjecutada(Mover(abajo),s) $\land$ posicion(x, SumarPosicion(y,1),s))
\newline $\lor$
\newline (AccionEjecutada(Mover(derecha),s) $\land$ posicion(SumarPosicion(x,-1),y,s))
\newline $\lor$
\newline (AccionEjecutada(Mover(izquierda),s) $\land$ posicion(SumarPosicion(x,1),y,s))
\newline $\lor$
\newline ($\lnot$AccionEjecutada(Mover(dirección),s) $\land$ posición(x,y,s))

% celdaVacía

\Nota{Agregamos los axiomas de estado sucesor para celdaVacia.}

\item $\forall x,y$: celdaVacia(x,y,s+1) $\Leftrightarrow$ (celdaVacia(x,y,s))

\item $\forall x,y$: celdaVacia(x,y,s+1) $\Leftrightarrow$
\newline (posicion(x,y,s) $\land$ hayComida(x,y,s) $\land$ AccionEjecutada(Comer,s))

\item $\forall x,y$: celdaVacia(x,y,s+1) $\Leftrightarrow$
\newline (posicion(x,y,s) $\land$ hayEnemigo(x,y,s) $\land$ AccionEjecutada(Pelear,s))

% hayComida

\Nota{Para hayComida, contemplamos el caso en que el agente se ubica en una
posición coincidente a la de un alimento (primer caso), y para los otros
alimentos conocidos pero en otras posiciones distintas a la del agente.}

\item $\forall x,y$: hayComida(x,y,s+1) $\Leftrightarrow$
\newline (posicion(x,y,s) $\land$ hayComida(x,y,s) $\land$ $\lnot$AccionEjecutada(Comer,s))

\item $\forall x,y$: hayComida(x,y,s+1) $\Leftrightarrow$
\newline (posicion(x1,y1,s) $\land$ (x $\ne$ x1 $\lor$ y $\ne$ y1) $\land$ hayComida(x,y,s))

% hayEnemigo

\Nota{Aquí vale la misma nota que hicimos para hayComida.}

\item $\forall x,y$: hayEnemigo(x,y,s+1) $\Leftrightarrow$
\newline (posicion(x,y,s) $\land$ hayEnemigo(x,y,s) $\land$ $\lnot$AccionEjecutada(Pelear,s))

\item $\forall x,y$: hayEnemigo(x,y,s+1) $\Leftrightarrow$
\newline (posicion(x1,y1,s) $\land$ (x $\ne$ x1 $\lor$ y $\ne$ y1) $\land$ hayEnemigo(x,y,s))

\end{itemize}

