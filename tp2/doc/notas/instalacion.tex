\subsection{Cómo compilar el código y ejecutarlo}

\subsubsection{En GNU/Linux}
\begin{itemize}

\item Crear un proyecto Java. Debe utilizarse Java 5.0.

\item Importar en el proyecto antes creado el archivo \archivo{tp2-ia.zip}
(General/Archive file).  Cuando pregunte por sobreescribir los archivos
\archivo{.project} y \archivo{.classpath} responder afirmativamente.

\item Hacer click derecho en la carpeta del proyecto, ir al menú \emph{Run As},
y elegir la opción \emph{Run...}

\item En el panel de la izquierda, hacer doble click sobre \emph{Java Application}.
para crear una configuración nueva.

\item En la solapa \emph{Main}, elegir como \emph{Main class} la clase \variable{tpsia.tp2.Main}.

\item En la solapa \emph{Arguments}, en la caja de texto \emph{Program arguments} colocar:
``logger.config'' (sin las comillas). Es el archivo de configuración de log4j. En la caja de texto \emph{VM arguments}
colocar:\newline``-Djava.library.path=\$\{SWI\_PROLOG\_INSTALL\_DIR\}/lib/i386-linux''.\newline
Reemplazar \$\{SWI\_PROLOG\_INSTALL\_DIR\} por el path absoluto donde fue instalado SWI-Prolog.

\item En la solapa \emph{Environment}, crear una nueva variable de entorno con el nombre ``SWI\_HOME\_DIR'',
y valor ``\$\{SWI\_PROLOG\_INSTALL\_DIR\}''.

\item Click en el botón \emph{Apply}.

\item Click en el botón \emph{Run}. La salida muestra un dibujo de la visión del ambiente del pacman,
las decisiones que va tomando, su posición, energía, etc.

\end{itemize}

\subsubsection{En Windows}

\TODO{Falta}
