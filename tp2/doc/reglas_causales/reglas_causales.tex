\section{Reglas causales}

\begin{itemize}

\subsection{Conveniencia de moverse y pelear}
\item  energia(energíaActual,s)
$\land$ promedioPorMoverse(promedio,s) $\land$ (energíaActual + promedio) >~0
$\Rightarrow$ convieneMoverse(s)

\item hayEnemigo(x,y,s) $\land$ energia(energíaActual,s)
$\land$ promedioPorPelear(promedio,s) $\land$ (energíaActual + promedio) >~0
$\Rightarrow$ convienePelear(s)

\subsection{Conocimiento de las celdas}
\item celdaVacía(x,y,s) $\Rightarrow$
 conoce(x,y,s)
 
\item hayEnemigo(x,y,s) $\Rightarrow$
 conoce(x,y,s)

\item hayComida(x,y,s) $\Rightarrow$
 conoce(x,y,s)

\subsection{Celdas adyacentes}
\item posicion(x,y,s) $\Rightarrow$ adyacente(SumarPosicion(x,1),y,derecha,s)

\item posicion(x,y,s) $\Rightarrow$ adyacente(SumarPosicion(x,-1),y,izquierda,s)

\item posicion(x,y,s) $\Rightarrow$ adyacente(x,SumarPosicion(y,1),arriba,s)

\item posicion(x,y,s) $\Rightarrow$ adyacente(x,SumarPosicion(y,-1),abajo,s)

\subsection{Cálculo de los promedios}
\item AccionEjecutada(acción,s) $\land$ energia(e1,s) $\land$ energia(e0,s-1)
$\Rightarrow$
     datosEnergia(acción,[(e0 - e1)|energias])
     
\item sumatoria(lista,suma)$\land$ cantidad(lista,c)$\land$ prom = (suma/c)
\newline
       $\Rightarrow$ promedio(lista,prom)

\item  datosEnergia(Comer,energias) $\land$ 
 promedio(energias,prom) $\Rightarrow$
 promedioPorComer(prom,s)
       
\item  datosEnergia(Pelear,energias) $\land$ 
 promedio(energias,prom) $\Rightarrow$
 promedioPorPelear(prom,s)

\item  datosEnergia(Mover,energias) $\land$ 
 promedio(energias,prom) $\Rightarrow$
 promedioPorMoverse(prom,s)

\subsection{Reglas de utilidad para evaluar los objetivos}
\FIXME{La regla debajo está DEPRECATED}
\item posicion(x,y,s) $\Rightarrow$
 conoce(SumarPosicion(x,-1,s),y,s) $\land$  
 conoce(SumarPosicion(x,1,s),y,s) $\land$ 
 conoce(x,SumarPosicion(y,-1,s),s) $\land$ 
 conoce(x,SumarPosicion(y,1,s),s) 

\TODO{Elegir si definimos esto por extensión o con un cuantificador}
\item conoce(1,1,s) $\land$ conoce(1,2,s) $\land$ 
 conoce(1,3,s) $\land$ conoce(1,4,s) $\land$ 
 \newline 
 conoce(2,1,s) $\land$ conoce(2,2,s) $\land$ 
 conoce(2,3,s) $\land$ conoce(2,4,s) $\land$ 
 \newline 
 conoce(3,1,s) $\land$ conoce(3,2,s) $\land$ 
 conoce(3,3,s) $\land$ conoce(3,4,s) $\land$ 
 \newline 
 conoce(4,1,s) $\land$ conoce(4,2,s) $\land$ 
 conoce(4,3,s) $\land$ conoce(4,4,s) 
 \newline
 $\Rightarrow$ conoceTodo(s)

\FIXME{Las tres reglas debajo están DEPRECATED}
\item celdaVacia(1,1,s) $\land$ celdaVacia(1,2,s) $\land$ 
 celdaVacia(1,3,s) $\land$ celdaVacia(1,4,s) $\land$ 
 \newline 
 celdaVacia(2,1,s) $\land$ celdaVacia(2,2,s) $\land$ 
 celdaVacia(2,3,s) $\land$ celdaVacia(2,4,s) $\land$
 \newline  
 celdaVacia(3,1,s) $\land$ celdaVacia(3,2,s) $\land$ 
 celdaVacia(3,3,s) $\land$ celdaVacia(3,4,s) $\land$ 
 \newline 
 celdaVacia(4,1,s) $\land$ celdaVacia(4,2,s) $\land$ 
 celdaVacia(4,3,s) $\land$ celdaVacia(4,4,s) 
 $\Rightarrow$ tableroVacio(s)
 
\item tableroVacio(s) $\Rightarrow$ noHayEnemigosVivos(s) 
\item tableroVacio(s) $\Rightarrow$ noHayComida(s)

\TODO{Ver como definimos esto. Según la profesora es mejor preguntar por cosas
que existen, para que quede más fácil de entender. Sin embargo, los dos predicados
de abajo funcionarían perfectamente me parece. Ver si avanzamos en pro a un mejor
entendimiento o a un planteo más... ¿elegante?}
\item !hayEnemigo(x,y,s) $\Rightarrow$ noHayEnemigosVivos(s) 
\item !hayComida(x,y,s) $\Rightarrow$ noHayComida(s)

\subsection{Planteo de los objetivos}
\item conoceTodo(s) $\Rightarrow$ condicionUnoObjetivo(s)

\TODO{La profesora planteó que podríamos definir esto como convieneMoverse([si|no],s)}
\item ! convieneMoverse(s) $\Rightarrow$ condicionUnoObjetivo(s)

\TODO{Yo propuse otra regla para la condición dos. Ver si va.}
\item noHayComida(s) $\Rightarrow$ condicionDosObjetivo(s)
\item noHayEnemigosVivos(s) $\Rightarrow$ condicionTresObejtivo(s)
\item ! convienePelear(s) $\Rightarrow$ condicionTresObejtivo(s)

\item  condicionUnoObjetivo(s) $\land$ condicionDosObjetivo(s)
$\land$ condicionTresObejtivo(s) $\Rightarrow$ cumplioObjetivo(s)

\end{itemize}


