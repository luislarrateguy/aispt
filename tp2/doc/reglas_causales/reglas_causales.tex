\section{Reglas causales}

\begin{itemize}

\subsection{Conveniencia de moverse y pelear}
\item  energia(energíaActual,s)
$\land$ promedioPorMoverse(promedio,s) $\land$ (energíaActual + promedio) >~0
$\Rightarrow$ convieneMoverse(s)

\item hayEnemigo(x,y,s) $\land$ energia(energíaActual,s)
$\land$ promedioPorPelear(promedio,s) $\land$ (energíaActual + promedio) >~0
$\Rightarrow$ convienePelear(s)

\subsection{Conocimiento de las celdas}
\item celdaVacía(x,y,s) $\Rightarrow$
 conoce(x,y,s)
 
\item hayEnemigo(x,y,s) $\Rightarrow$
 conoce(x,y,s)

\item hayComida(x,y,s) $\Rightarrow$
 conoce(x,y,s)

\subsection{Celdas adyacentes}
\item posicion(x,y,s) $\Rightarrow$ adyacente(SumarPosicion(x,1),y,derecha,s)

\item posicion(x,y,s) $\Rightarrow$ adyacente(SumarPosicion(x,-1),y,izquierda,s)

\item posicion(x,y,s) $\Rightarrow$ adyacente(x,SumarPosicion(y,1),arriba,s)

\item posicion(x,y,s) $\Rightarrow$ adyacente(x,SumarPosicion(y,-1),abajo,s)

\subsection{Cálculo de los promedios}
\item AccionEjecutada(acción,s) $\land$ energia(e1,s) $\land$ energia(e0,s-1)
$\Rightarrow$
     datosEnergia(acción,[(e0 - e1)|energias])
     
\item sumatoria(lista,suma)$\land$ cantidad(lista,c)$\land$ prom = (suma/c)
\newline
       $\Rightarrow$ promedio(lista,prom)

\item  datosEnergia(Comer,energias) $\land$ 
 promedio(energias,prom) $\Rightarrow$
 promedioPorComer(prom,s)
       
\item  datosEnergia(Pelear,energias) $\land$ 
 promedio(energias,prom) $\Rightarrow$
 promedioPorPelear(prom,s)

\item  datosEnergia(Mover,energias) $\land$ 
 promedio(energias,prom) $\Rightarrow$
 promedioPorMoverse(prom,s)

\subsection{Reglas de utilidad para evaluar los objetivos}

\TODO{Elegir si definimos esto por extensión o con un cuantificador.
\newline 
Nacho says: me da lo mismo. Al momento de la implementacion lo hacemos por 
extension por facilidad, y aca podriamos hacerlo por cuantificador para 
expresarlo.}
\item conoce(1,1,s) $\land$ conoce(1,2,s) $\land$ 
 conoce(1,3,s) $\land$ conoce(1,4,s) $\land$ 
 \newline 
 conoce(2,1,s) $\land$ conoce(2,2,s) $\land$ 
 conoce(2,3,s) $\land$ conoce(2,4,s) $\land$ 
 \newline 
 conoce(3,1,s) $\land$ conoce(3,2,s) $\land$ 
 conoce(3,3,s) $\land$ conoce(3,4,s) $\land$ 
 \newline 
 conoce(4,1,s) $\land$ conoce(4,2,s) $\land$ 
 conoce(4,3,s) $\land$ conoce(4,4,s) 
 \newline
 $\Rightarrow$ conoceTodo(s)
 
\Nota{Esto no deja de ser cierto, que si todas estan vacias luego no hay
enemigos ni comida. No hace falta discriminar cual ocurre. No va afectar al 
objetivo, ni va a mejorar como se comporta el agente, ya que es EXCELENTE comer.
}
\item $\forall$ celdaVacia(X,Y,s) 
 $\Rightarrow$ noHayEnemigosVivos(s) $\land$ noHayComida(s)

\subsection{Planteo de los objetivos}
\item conoceTodo(s) $\Rightarrow$ condicionUnoObjetivo(s)

\item convieneMoverse(`no',s) $\Rightarrow$ condicionUnoObjetivo(s)

\TODO{Yo propuse otra regla para la condición dos. Ver si va.}
\item noHayComida(s) $\Rightarrow$ condicionDosObjetivo(s)
\item noHayEnemigosVivos(s) $\Rightarrow$ condicionTresObejtivo(s)
\item convienePelear(`no',s) $\Rightarrow$ condicionTresObejtivo(s)

\item  condicionUnoObjetivo(s) $\land$ condicionDosObjetivo(s)
$\land$ condicionTresObejtivo(s) $\Rightarrow$ cumplioObjetivo(s)

\end{itemize}


