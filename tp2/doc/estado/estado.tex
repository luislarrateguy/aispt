\section{Representación del estado}

\begin{itemize}

\item \textbf{posicion(x,y,s)}: la posicion del pacman en la situación $s$ es
la columna $x$ y la fila $y$.

\item \textbf{energía(e,s)}: el pacman tiene una energía igual a $e$ en la
situación $s$.

\item \textbf{conoce(x,y,s)}: indica que el agente conoce la celda ubicada
en la columna $x$ y fila $y$.

\item \textbf{hayEnemigo(x,y,s)}: hay un enemigo en la columna $x$ fila $y$ en
la situación $s$.

\item \textbf{hayComida(x,y,s)}: hay comida en la columna $x$ fila $y$ en la
situación $s$.

\item \textbf{celdaVacía(x,y,s)}: la celda en la columna $x$ fila $y$ está
vacía.

\item \textbf{promedioPorPelear(prom,s)}: promedio energía perdida por pelear.
El valor de los promedios es positivo si implica una ganancia y negativo en
caso contrario.

\item \textbf{promedioPorAvanzar(prom,s)}: promedio energía perdida por
avanzar. El valor de los promedios es positivo si implica una ganancia y
negativo en caso contrario.

\item \textbf{promedioPorComer(prom,s)}: promedio energía ganada por comer. El
valor de los promedios es positivo si implica una ganancia y negativo en caso
contrario.

\item \textbf{datosEnergia(Pelear,[],s)}: Almacena una lista con las diferencias
entre la energía percibida y la actual cuando la acción fue Pelear, para después
sacar un promedio.

\item \textbf{datosEnergia(Mover,[],s)}: Ídem, pero para la acción Mover.

% DEPRECATED: Este dato no haría falta. Sí lo utilizabamos en búsqueda, pero
% aquí no hace falta.
\item \textbf{datosEnergia(Comer,[],s)}: Ídem, pero para la acción Comer.

\item \textbf{adyacentes(xa,ya,dirección,s)}: devuelve en $xa$ y en $ya$ la
posición de una celda adyacente a la actual ($posicion(x,y,s)$). $direccion$
devuelve los mismos valores que acepta la acción \emph{Mover}.

\item \textbf{AccionEjecutada(acción,s)}: Indica que ``acción'' fue ejecutada
en la situación \emph{s}.

\end{itemize}
